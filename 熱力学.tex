\documentclass[11pt,a4paper]{jsarticle}
\begin{document}
\title{\Huge 熱力学}
\author{澤田大地}
\maketitle

\tableofcontents
\newpage

\section{熱平衡状態と温度}
熱力学の基本的な考え方や、熱力学の対象となる
系の基本的な性質についてまとめる

 \subsection{熱力学第0法則}
 『物体Aと物体B、物体Bと物体Cがそれぞれ熱平衡状態
 にあるとき、物体Aと物体Cは熱平衡状態である』こと
 を熱力学第0法則という。(経験則)
 \subsection{温度の定義}
 水の凝固点を0℃、沸点を100℃と定義するものを
 経験的温度といい、t[℃]とする。

 気体の圧力が0になる時の経験的温度は全物質で
 等しく、この時の経験的温度t=-273.15℃を用い、

 θ[K]≡t[℃]+273.15

 と(経験的)理想気体絶対温度θ[K]を定義する。
 \subsection{熱の定義}
 熱量は状態量ではなく非状態量である。
 ここで、ある状態からある状態に至る際に、
 そこに至る経緯が変わらないものを状態量、
 変わるものを非状態量という。
 (ex 等圧、等温、断熱変化によって系から
 取り出せる熱量や仕事はそれぞれ変化する
 ためこの2つは非状態量である)
\subsection{状態方程式}
熱平衡状態において

%理想気体の状態方程式
%ファンデルワールス気体の状態方程式
\subsection{示量変数と示強変数}


\end{document}
